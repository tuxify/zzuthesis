%%================================================
%% Filename: chap04.tex
%% Encoding: UTF-8
%% Author: Yuan Xiaoshuai - yxshuai@gmail.com
%% Created: 2012-04-28 00:15
%% Last modified: 2016-08-28 21:08
%%================================================
\chapter{常用排版示例}
\label{cha:example}

\section{插图示例}
\label{sec:figure}

强烈推荐《\LaTeXe 插图指南》!关于子图形的使用细节请参看 \textbf{subfig} 的说
明文档。

\subsection{\LaTeX~中推荐使用的图片格式}

在~\LaTeX~中应用最多的图片格式是~EPS(Encapsulated PostScript)格式,它是一种
专用的打印机描述语言,常用于印刷或打印输出。EPS~格式图片可通过多种方式生成,
如果使用命令行界面,则可以用 ImageMagick,其可将其它格式图片转换为~EPS~格式图
片,同时还可以锐化图片,使图片的局部清晰一些。

ImageMagick 包含多个命令行程序,其中最常用的是 \texttt{convert}。\\
\verb|convert [可选参数] 原文件名.原扩展名 新文件名.eps|

除此之外,一些文字处理软件和科学计算软件也支持生成~EPS~格式的文件,可用“另存
为”功能查看某款软件是否能够将图片以~EPS~格式的形式保存。

\subsection{插入单幅图形}

插图通常需要占据大块空白,所以在文字处理软件中用户经常需要调整插图的位置。
\LaTeX~有一个 \texttt{figure} 环境可以自动完成这样的任务,这种自动调整位置的环
境称作浮动环境(float),之后还会介绍表格浮动环境。

单张图片插入形式如图~\ref{golfer1}~所示。
\begin{figure}[htbp]
\centering
\includegraphics[width = 0.3\textwidth]{golfer}
\caption{打高尔夫球的人}
\label{golfer1}
\end{figure}

\begin{code}
\begin{figure}[htbp]
\centering
\includegraphics[width=0.3\textwidth]{golfer}
\caption{打高尔夫球的人}
\label{golfer1}
\end{figure}
\end{code}

上述代码中,\verb|[htbp]| 选项用来指定插图排版的理想位置,这几个字母分别代表
~here、top、bottom、float page,也就是固定位置、页顶、页尾、单独的浮动页。可
以使用这几个字母的任意组合,一般不推荐单独使用 \verb|[h]|。如果要强制固定浮动
图形的位置,还可以用 \textbf{float} 宏包的 \verb|[H]| 选项。

\verb|\centering| 用来使插图居中,\verb|\caption| 命令设置插图标题,\LaTeX~会
自动给浮动环境的标题加上编号。注意 \verb|label| 应放在 \verb|caption| 之后,否
则引用时指向的是前一个插图。

\subsection{插入多幅图形}

\subsubsection*{并排摆放,共享标题}

当我们需要两幅图片并排摆放,并共享标题时,可以在 \texttt{figure} 环境中使用两个
\verb|\includegraphics| 命令,如图~\ref{fig:golfer2}~所示。

\begin{code}
\begin{figure}[htbp]
\centering
\includegraphics[width=0.3\textwidth]{left}
\hspace{36pt}
\includegraphics[width=0.3\textwidth]{right}
\caption{反清复明}
\label{fig:golfer2}
\end{figure}
\end{code}

\begin{figure}[htbp]
\centering
\includegraphics[width=0.3\textwidth]{left}
\hspace{36pt}
\includegraphics[width=0.3\textwidth]{right}
\caption{反清复明}
\label{fig:golfer2}
\end{figure}

\subsubsection*{并排摆放,各有标题}

如果想要两幅并排的图片各有自己的标题,可以在 \texttt{figure} 环境中使用两个
 \texttt{minipage} 环境,每个环境里插入一个图,如图~\ref{fig:golfer3}~和
~\ref{fig:golfer4}~所示。

\begin{code}
\begin{figure}[htbp]
\centering
\begin{minipage}[t]{0.3\textwidth}
    \centering
    \includegraphics[width=\textwidth]{left}
    \caption{清明}
    \label{fig:golfer3}
\end{minipage}
\hspace{36pt}
\begin{minipage}[t]{0.3\textwidth}
    \centering
    \includegraphics[width=\textwidth]{right}
    \caption{反复}
    \label{fig:golfer4}
\end{minipage}
\end{figure}
\end{code}

\begin{figure}[htbp]
\centering
\begin{minipage}[t]{0.3\textwidth}
    \centering
    \includegraphics[width=\textwidth]{left}
    \caption{清明}
    \label{fig:golfer3}
\end{minipage}
\hspace{36pt}
\begin{minipage}[t]{0.3\textwidth}
    \centering
    \includegraphics[width=\textwidth]{right}
    \caption{反复}
    \label{fig:golfer4}
\end{minipage}
\end{figure}

\subsubsection*{并排摆放,共享标题,各有子标题}

如果想要两幅并排的图片共享一个标题,并各有自己的子标题,可以使用
\textbf{subfig} 宏包提供的 \verb|\subfloat| 命令,如图~\ref{fig:subfig_a}~和
~\ref{fig:subfig_b}~所示。

\verb|\subfloat| 命令缺少宽度参数。虽然我们可以用 \verb|\hspace| 命令调整子图
的距离,子标题却只能和子图本身一样宽,就会出现折行。

\begin{code}
\begin{figure}[htbp]
\centering
\subfloat[清明]{
    \label{fig:subfig_a}
    \includegraphics[width=0.3\textwidth]{left}
}
\hspace{36pt}
\subfloat[反复]{
    \label{fig:subfig_b}
    \includegraphics[width=0.3\textwidth]{right}
}
\caption{反清复明}
\end{figure}
\end{code}

每个子图可以有各自的引用,就象这个样子:\ref{fig:subfig_a}、
\ref{fig:subfig_b}。

\begin{figure}[htbp]
\centering
\subfloat[清明]{
    \label{fig:subfig_a}
    \includegraphics[width=0.3\textwidth]{left}
}
\hspace{36pt}
\subfloat[反复]{
    \label{fig:subfig_b}
    \includegraphics[width=0.3\textwidth]{right}
}
\caption{反清复明}
\end{figure}

\subsubsection*{改进的子图方法}

为了避免子标题折行,我们可以在 \verb|\subfloat| 里再嵌套个 \verb|minipage|,因
为后者是有宽度的,如图~\ref{fig:improved_subfig_a}~和
~\ref{fig:improved_subfig_b}~所示。

\begin{code}
\begin{figure}[htbp]
\centering
\subfloat[清明]{
\label{fig:improved_subfig_a}
\begin{minipage}[t]{0.3\textwidth}
    \centering
    \includegraphics[width=\textwidth]{left}
\end{minipage}
}
\hspace{36pt}
\subfloat[反复]{
\label{fig:improved_subfig_b}
\begin{minipage}[t]{0.3\textwidth}
    \centering
    \includegraphics[width=\textwidth]{right}
\end{minipage}
}
\caption{反清复明}
\end{figure}
\end{code}

\begin{figure}[htbp]
\centering
\subfloat[清明]{
\label{fig:improved_subfig_a}
\begin{minipage}[t]{0.3\textwidth}
    \centering
    \includegraphics[width=\textwidth]{left}
\end{minipage}
}
\hspace{36pt}
\subfloat[反复]{
\label{fig:improved_subfig_b}
\begin{minipage}[t]{0.3\textwidth}
    \centering
    \includegraphics[width=\textwidth]{right}
\end{minipage}
}
\caption{反清复明}
\end{figure}

\section{表格示例}
\label{sec:table}

模板中关于表格的宏包有四个:\textbf{tabularx}、\textbf{multirow}、
%\textbf{slashbox}、\textbf{array}
\textbf{longtable} 和\textbf{booktabs}。长表格还可以用\textbf{supertabular},
可以方便地在表格下方加入脚注。

\subsection{普通表格的绘制方法}

\texttt{table} 环境是一个将表格嵌入文本的浮动环境,其标题和交叉引用的用法类似
于上一节提到的图形浮动环境 \texttt{figure}。该环境提供了最简单的表格功能。它
用 \verb|\hline| 命令表示横线,\verb&|& 表示竖线;用 \verb|&| 来分列,用
\verb|\\| 来换行;每列可以采用居中、居左、居右等横向对齐方式,分别用 \verb|l|
、\verb|c|、\verb|r| 来表示。

科技文献中常使用三线表格,因此需要调用 \textbf{booktabs} 宏包,三条横线就分别
用 \verb|\toprule|、\verb|\midrule|、\verb|\bottomrule| 等命令表示。其标准格
式如表~\ref{table1}~所示。

\begin{table}[htbp]
\caption{文献类型和标识代码}
\label{table1}
\centering
\begin{tabular}{cccc}
\toprule
文献类型 & 标识代码 & 文献类型 & 标识代码\\
\midrule
普通图书 & M &  会议录 & C\\
汇编 & G & 报纸 & N\\
期刊 & J & 学位论文 & D\\
报告 & R & 标准 & S\\
专利 & P & 数据库 & DB\\
计算机程序 & CP & 电子公告 & EB\\
\bottomrule
\end{tabular}
\end{table}

绘制该表格的代码如下:
\begin{code}
\begin{table}[htbp]
\caption{表格标题}
\label{标签名}
\centering
\begin{tabular}{cc...c}
\toprule
表头第1个格   & 表头第2个格   & ... & 表头第n个格  \\
\midrule
表中数据(1,1) & 表中数据(1,2) & ... & 表中数据(1,n)\\
表中数据(2,1) & 表中数据(2,2) & ... & 表中数据(2,n)\\
...................................................\\
表中数据(m,1) & 表中数据(m,2) & ... & 表中数据(m,n)\\
\bottomrule
\end{tabular}
\end{table}
\end{code}

\subsection{长表格的绘制方法}

长表格是当表格在当前页排不下而需要转页接排的情况下所采用的一种表格环境。若长
表格仍按照普通表格的绘制方法来获得,其所使用的 \verb|table| 浮动环境无法实现
表格的换页接排功能,表格下方过长部分会排在表格第1页的页脚以下。

% \textbf{supertabular} 宏包提供一个 \texttt{supertabular} 环境,是对
% \texttt{tabular} 环境的扩充。它能不断地计算表格长度,当排版到页面底部时,自动
% 结束 \texttt{tabular} 环境,而在下一页再自动生成一个新的 \texttt{tabular} 环境
% ,将剩余表格放入其中。使用该宏包排版长表格时,要用所提供的生成命令专门设计表头
% ,具体方法如下:
% 
% \begin{code}
% \begin{center}
% \tablefirsthead{\toprule 表头第1个格 & 表头第2个格 & ... & 表头第n个格\\
  % \midrule}
% \tablehead{\multicolumn{n}{c}{表~\thetable(续表)}\\
  % \toprule 表头第1个格 & 表头第2个格 & ... & 表头第n个格\\ \midrule}
% \tabletail{\bottomrule
  % \multicolumn{n}{r}{续下页}\\}
% \tablelasttail{\bottomrule}
% \tablecaption{表格标题}\label{标签名}
% \begin{supertabular}{cc...c}
% 表中数据(1,1) & 表中数据(1,2) & ... & 表中数据(1,n)\\
% 表中数据(2,1) & 表中数据(2,2) & ... & 表中数据(2,n)\\
% ...................................................\\
% 表中数据(m,1) & 表中数据(m,2) & ... & 表中数据(m,n)\\
% \end{supertabular}
% \end{center}
% \end{code}
% 
% \verb|\tablehead| 命令描述的是第2页及其之后各页的标题或表头;
% \verb|\tablefirsthead| 命令描述的是第1页的标题和表头,若无此命令,则第1页
% 的表头和标题由 \verb|\tablehead| 命令确定;同理,\verb|\tabletail| 命令描述
% 的是除最后一页之外每页的表格底部内容;\verb|\tablelasttail| 之前的文字描述的是
% 最后一页的表格底部内容,若无此命令,则最后一页表格底部内容由 \verb|\tabletail|
% 命令确定。所绘制的长表格的格式如表~\ref{table2}~所示。
\begin{longtable}{l@{\hspace{6.5mm}}l@{\hspace{5.5mm}}l}
\multicolumn{3}{c}{续表~\thetable\hskip1em 中国省级行政单位一览}\\
\toprule 名称 & 简称 & 省会或首府  \\ \midrule
\endhead
\caption{中国省级行政单位一览}
\label{table2}\\
\toprule 名称 & 简称 & 省会或首府  \\ \midrule
\endfirsthead
\bottomrule
\multicolumn{3}{r}{续下页}\\
\endfoot
\bottomrule
\endlastfoot
北京市 & 京 & 北京\\
天津市 & 津 & 天津\\
河北省 & 冀 & 石家庄市\\
山西省 & 晋 & 太原市\\
内蒙古自治区 & 蒙 & 呼和浩特市\\
辽宁省 & 辽 & 沈阳市\\
吉林省 & 吉 & 长春市\\
黑龙江省 & 黑 & 哈尔滨市\\
上海市 & 沪/申 & 上海\\
江苏省 & 苏 & 南京市\\
浙江省 & 浙 & 杭州市\\
安徽省 & 皖 & 合肥市\\
福建省 & 闽 & 福州市\\
江西省 & 赣 & 南昌市\\
山东省 & 鲁 & 济南市\\
河南省 & 豫 & 郑州市\\
湖北省 & 鄂 & 武汉市\\
湖南省 & 湘 & 长沙市\\
广东省 & 粤 & 广州市\\
广西壮族自治区 & 桂 & 南宁市\\
海南省 & 琼 & 海口市\\
重庆市 & 渝 & 重庆\\
四川省 & 川/蜀 & 成都市\\
贵州省 & 黔/贵 & 贵阳市\\
云南省 & 云/滇 & 昆明市\\
西藏自治区 & 藏 & 拉萨市\\
陕西省 & 陕/秦 & 西安市\\
甘肃省 & 甘/陇 & 兰州市\\
青海省 & 青 & 西宁市\\
宁夏回族自治区 & 宁 & 银川市\\
新疆维吾尔自治区 & 新 & 乌鲁木齐市\\
香港特别行政区 & 港 & 香港\\
澳门特别行政区 & 澳 & 澳门\\
台湾省 & 台 & 台北市\\
北京市 & 京 & 北京\\
天津市 & 津 & 天津\\
河北省 & 冀 & 石家庄市\\
山西省 & 晋 & 太原市\\
内蒙古自治区 & 蒙 & 呼和浩特市\\
辽宁省 & 辽 & 沈阳市\\
吉林省 & 吉 & 长春市\\
黑龙江省 & 黑 & 哈尔滨市\\
上海市 & 沪/申 & 上海\\
江苏省 & 苏 & 南京市\\
浙江省 & 浙 & 杭州市\\
安徽省 & 皖 & 合肥市\\
福建省 & 闽 & 福州市\\
江西省 & 赣 & 南昌市\\
山东省 & 鲁 & 济南市\\
河南省 & 豫 & 郑州市\\
湖北省 & 鄂 & 武汉市\\
湖南省 & 湘 & 长沙市\\
广东省 & 粤 & 广州市\\
广西壮族自治区 & 桂 & 南宁市\\
海南省 & 琼 & 海口市\\
重庆市 & 渝 & 重庆\\
四川省 & 川/蜀 & 成都市\\
贵州省 & 黔/贵 & 贵阳市\\
云南省 & 云/滇 & 昆明市\\
西藏自治区 & 藏 & 拉萨市\\
陕西省 & 陕/秦 & 西安市\\
甘肃省 & 甘/陇 & 兰州市\\
青海省 & 青 & 西宁市\\
宁夏回族自治区 & 宁 & 银川市\\
新疆维吾尔自治区 & 新 & 乌鲁木齐市\\
香港特别行政区 & 港 & 香港\\
澳门特别行政区 & 澳 & 澳门\\
台湾省 & 台 & 台北市\\
\end{longtable}
% \begin{center}
% \tablefirsthead{\toprule 名称 & 简称 & 省会或首府  \\ \midrule}
% \tablehead{\multicolumn{3}{c}{续表~\thetable\hskip1em 中国省级行政单位一览}\\
% \toprule 名称 & 简称 & 省会或首府  \\ \midrule}
% \tabletail{\bottomrule
% \multicolumn{3}{r}{续下页}\\}
% \tablelasttail{\bottomrule}
% \tablecaption{中国省级行政单位一览}
% \label{table2}
% \begin{supertabular}{l@{\hspace{6.5mm}}l@{\hspace{5.5mm}}l}
% 北京市 & 京 & 北京\\
% 天津市 & 津 & 天津\\
% 河北省 & 冀 & 石家庄市\\
% 山西省 & 晋 & 太原市\\
% 内蒙古自治区 & 蒙 & 呼和浩特市\\
% 辽宁省 & 辽 & 沈阳市\\
% 吉林省 & 吉 & 长春市\\
% 黑龙江省 & 黑 & 哈尔滨市\\
% 上海市 & 沪/申 & 上海\\
% 江苏省 & 苏 & 南京市\\
% 浙江省 & 浙 & 杭州市\\
% 安徽省 & 皖 & 合肥市\\
% 福建省 & 闽 & 福州市\\
% 江西省 & 赣 & 南昌市\\
% 山东省 & 鲁 & 济南市\\
% 河南省 & 豫 & 郑州市\\
% 湖北省 & 鄂 & 武汉市\\
% 湖南省 & 湘 & 长沙市\\
% 广东省 & 粤 & 广州市\\
% 广西壮族自治区 & 桂 & 南宁市\\
% 海南省 & 琼 & 海口市\\
% 重庆市 & 渝 & 重庆\\
% 四川省 & 川/蜀 & 成都市\\
% 贵州省 & 黔/贵 & 贵阳市\\
% 云南省 & 云/滇 & 昆明市\\
% 西藏自治区 & 藏 & 拉萨市\\
% 陕西省 & 陕/秦 & 西安市\\
% 甘肃省 & 甘/陇 & 兰州市\\
% 青海省 & 青 & 西宁市\\
% 宁夏回族自治区 & 宁 & 银川市\\
% 新疆维吾尔自治区 & 新 & 乌鲁木齐市\\
% 香港特别行政区 & 港 & 香港\\
% 澳门特别行政区 & 澳 & 澳门\\
% 台湾省 & 台 & 台北市\\
% \end{supertabular}
% \end{center}

表格~\ref{table2}~第~2~页的标题和表头是通过代码自动添加上去的。若表格在页面中
的竖直位置发生了变化,其在第~2~页及之后各页的标题和表头位置能够始终处于各页的
最顶部,无需调整。

\subsection{列宽可调表格的绘制方法}

论文中能用到列宽可调表格的情况共有两种,一种是当插入的表格某一单元格内容过长
以至于一行放不下的情况,另一种是当对公式中首次出现的物理量符号进行注释的情况
,这两种情况都需要调用 \textbf{tabularx} 宏包。下面将分别对这两种情况下可调表格的绘制
方法进行阐述。

\subsubsection{表格内某单元格内容过长的情况}

首先给出这种情况下的一个例子如表~\ref{table3}~所示。

\begin{table}[htbp]
\caption{最小的三个正整数的英文表示法}
\label{table3}
\begin{tabularx}{\textwidth}{llX}
\toprule
Value & Name & Alternate names, and names for sets of the given size\\
\midrule
1 & One & ace, single, singleton, unary, unit, unity\\
2 & Two & binary, brace, couple, couplet, distich, deuce, double, doubleton, duad, duality, duet, duo, dyad, pair, snake eyes, span, twain, twosome, yoke\\
3 & Three & deuce-ace, leash, set, tercet, ternary, ternion, terzetto, threesome, tierce, trey, triad, trine, trinity, trio, triplet, troika, hat-trick\\\bottomrule
\end{tabularx}
\end{table}

绘制该表格的代码如下:

\begin{code}
\begin{table}[htbp]
\caption{表格标题}
\label{标签名}
\begin{tabularx}{\textwidth}{l...X...l}
\toprule
表头第1个格   & ... & 表头第X个格   & ... & 表头第n个格  \\
\midrule
表中数据(1,1) & ... & 表中数据(1,X) & ... & 表中数据(1,n)\\
表中数据(2,1) & ... & 表中数据(2,X) & ... & 表中数据(2,n)\\
.........................................................\\
表中数据(m,1) & ... & 表中数据(m,X) & ... & 表中数据(m,n)\\
\bottomrule
\end{tabularx}
\end{table}
\end{code}

\texttt{tabularx} 环境共有两个必选参数:第1个参数用来确定表格的总宽度,这里取
为排版表格能达到的最大宽度——正文宽度 \verb|\textwidth|;第2个参数用来确定每列
格式,其中标为 \verb|X| 的项表示该列的宽度可调,其宽度值由表格总宽度确定。标
为 \verb|X| 的列一般选为单元格内容过长而无法置于一行的列,这样使得该列内容能
够根据表格总宽度自动分行。若列格式中存在不止一个 \verb|X| 项,则这些标为
\verb|X| 的列的列宽相同,因此,一般不将内容较短的列设为 \verb|X| 。标为
\verb|X| 的列均为左对齐,因此其余列一般选为 \verb|l| (左对齐),这样可使得表格
美观,但也可以选为 \verb|c| 或 \verb|r|。

\subsubsection{对物理量符号进行注释的情况}

为使得对公式中物理量符号注释的转行与破折号“\pozhehao ”后第一个字对齐,此处最
好采用表格环境。此表格无任何线条,左对齐,且在破折号处对齐,一共有“式中”二字
、物理量符号和注释三列,表格的总宽度可选为文本宽度,因此应该采用
\texttt{tabularx} 环境。由该环境生成的对公式中物理量符号进行注释的公式如式
(\ref{eq:1})所示。

\begin{equation}
\label{eq:1}
\ddot{\bm{\rho}}-\frac{\mu}{R_t^3}\left(3\bm{R_t}\frac{\bm{R_t\rho}}{R_t^2}-\bm{\rho}\right)=\bm{a}
\end{equation}
\begin{flushleft}
\renewcommand\arraystretch{1.25}
\begin{tabularx}{\textwidth}{@{}>{\normalsize\rm}l@{\quad}>{\normalsize\rm}l@{\pozhehao }>{\normalsize\rm}X@{}}
式中& $\bm{\rho}$ &追踪飞行器与目标飞行器之间的相对位置矢量;\\
&  $\ddot{\bm{\rho}}$&追踪飞行器与目标飞行器之间的相对加速度;\\
&  $\bm{a}$   &推力所产生的加速度;\\
&  $\bm{R_t}$ & 目标飞行器在惯性坐标系中的位置矢量;\\
&  $\omega_{t}$ & 目标飞行器的轨道角速度;\\
&  $\bm{g}$ & 重力加速度,$=\frac{\mu}{R_{t}^{3}}\left(
3\bm{R_{t}}\frac{\bm{R_{t}\rho}}{R_{t}^{2}}-\bm{\rho}\right)=\omega_{t}^{2}\frac{R_{t}}{p}\left(
3\bm{R_{t}}\frac{\bm{R_{t}\rho}}{R_{t}^{2}}-\bm{\rho}\right)$,这里~$p$~是目标飞行器的轨道半通径。
\end{tabularx}\vspace{.5ex}%TODO : 注释内容自动转页接排
\end{flushleft}
% 式中\quad
% \parbox[t][][c]{2em}{
  % $\ddot{\bm{\rho}}$\\
  % $\bm{a}$\\
  % $\bm{R_t}$\\
  % $\omega_{t}$\\
% }
% \parbox[t][][c]{2em}{
  % \pozhehao\\
  % \pozhehao\\
  % \pozhehao\\
  % \pozhehao\\
% }
% \parbox[t][][c]{\textwidth-8em}{
 % 追踪飞行器与目标飞行器之间的相对加速度;\\
 % 推力所产生的加速度;\\
 % 目标飞行器在惯性坐标系中的位置矢量;\\
 % 目标飞行器的轨道角速度;\\}

% 由此方法生成的注释内容应紧邻待注释公式并置于其下方,因此不能将代码放入
% \texttt{table} 浮动环境中。但此方法不能实现自动转页接排,可能会在当前页剩余空
% 间不够时,全部移动到下一页而导致当前页出现很大空白。

其中生成注释部分的代码及其说明如下:
\begin{code}
\begin{tabularx}{\textwidth}{@{}l@{\quad}l@{\pozhehao}X@{}}
式中 & symbol-1 & symbol-1的注释内容;\\
     & symbol-2 & symbol-2的注释内容;\\
     .............................;\\
     & symbol-m & symbol-m的注释内容。
\end{tabularx}
\end{code}

\texttt{tabularx} 环境的第1个参数为正文宽度,第2个参数里面各个符号的意义为:
\begin{itemize}
\item 第1个@{}表示在“式中”二字左侧不插入任何文本,“式中”二字能够在正文中左对
齐,若无此项,则“式中”二字左侧会留出一定的空白;
\item \verb|@{\quad}| 表示在“式中”和物理量符号间插入一个空铅宽度的空白;
\item \verb|@{\pozhehao}| 实现插入破折号的功能;
\item 第2个 \verb|@{}| 表示在注释内容靠近正文右边界的地方能够实现右对齐。
\end{itemize}

% 若想获得绘制表格的更多信息,请参见网络上的~Tables in \LaTeXe: Packages and
% Methods~文档~http://www.tug.org/pracjourn/2007-1/mori/。 

% \begin{table}[htb]
  % \centering
  % \begin{minipage}[t]{0.8\linewidth} % 如果想在表格中使用脚注,minipage是个不错的办法
  % \caption[模板文件]{模板文件。如果表格的标题很长,那么在表格索引中就会很不美
    % 观,所以要像 chapter 那样在前面用中括号写一个简短的标题。这个标题会出现在索
    % 引中。}
  % \label{tab:template-files}
    % \begin{tabular*}{\linewidth}{lp{10cm}}
      % \toprule
      % {\hei 文件名} & {\hei 描述} \\
      % \midrule
      % zzuthesis.cls & 模板类文件。\\
      % zzuthesis.cfg & 模板配置文。\\
      % thubib.bst    & 参考文献 Bibtex 样式文件。\\
      % main.tex      & 主文档。\\
      % Makefile      & Linux 自动处理文件。\footnote{Linux系统自动编译处理。}\\
      % msmake      & Windows 自动处理文件。\footnote{Windows系统自动编译。}\\
      % data\textbackslash & 封面、摘要、符号说明、正文章节、简历、致谢等。\\
      % figures\textbackslash & 文档用到的图片。\\
      % ref\textbackslash & 参考文献数据库。\\
      % \bottomrule
    % \end{tabular*}
  % \end{minipage}
% \end{table}

% 表\ref{tab:template-files} 列举了本模板主要文件及其功能,还展示了如何在表格
% 中正确使用脚注。由于 \LaTeX{} 本身不支持在表格中使用 \verb|\footnote|,所以
% 不得不将表格放在小页中,而且最好将表格的宽度设置为小页的宽度,这样脚注看起
% 来才更美观。

% 我们经常会在表格下方标注数据来源,或者对表格里面的条目进行解释。前面的脚注
% 是一种不错的方法,如果你不喜欢脚注。那么完全可以在表格后面自己写注释,比如
% 表~\ref{tab:tabexamp1}。

% \begin{table}[ht]
  % \centering
  % \caption{复杂表格示例 1}
  % \label{tab:tabexamp1}
  % \begin{minipage}[t]{0.8\textwidth} 
    % \begin{tabularx}{\linewidth}{|l|X|X|X|X|}
      % \hline
 % \multirow{2}*{\backslashbox{x}{y}}  & \multicolumn{2}{c|}{First Half} & \multicolumn{2}{c|}{Second Half}\\\cline{2-5}
      % & 1st Qtr &2nd Qtr&3rd Qtr&4th Qtr \\ \hline
      % East$^{*}$ &   20.4&   27.4&   90&     20.4 \\
      % West$^{**}$ &   30.6 &   38.6 &   34.6 &  31.6 \\ \hline
    % \end{tabularx}\\[2pt]
    % \footnotesize 注:数据来源《\zzuthesis{} 使用手册》。\\
    % *:东部\\
    % **:西部
  % \end{minipage}
% \end{table}

% 此外,表~\ref{tab:tabexamp1} 同时还演示了另外两个功能:1)通过
% \textbf{tabularx} 的\texttt{|X|} 扩展实现表格自动放大;2)通过命令
% \verb|\backslashbox| 在表头部分插入反斜线。

% 浮动体的并排放置一般有两种情况:1)二者没有关系,为两个独立的浮动体;2)二
% 者隶属于同一个浮动体。对表格来说并排表格既可以像表~\ref{tab:parallel1}、表
% ~\ref{tab:parallel2} 使用小页环境,也可以如表~\ref{tab:subtable} 使用子表格
% 来做。

% \begin{table}[ht]
% \noindent\begin{minipage}{0.5\textwidth}
% \centering
% \caption{第一个并排子表格}
% \label{tab:parallel1}
% \begin{tabular}{p{2cm}p{2cm}}
% \toprule[1.5pt]
% 111 & 222 \\\midrule[1pt]
% 222 & 333 \\\bottomrule[1.5pt]
% \end{tabular}
% \end{minipage}
% \begin{minipage}{0.5\textwidth}
% \centering
% \caption{第二个并排子表格}
% \label{tab:parallel2}
% \begin{tabular}{p{2cm}p{2cm}}
% \toprule[1.5pt]
% 111 & 222 \\\midrule[1pt]
% 222 & 333 \\\bottomrule[1.5pt]
% \end{tabular}
% \end{minipage}
% \end{table}

% \begin{table}
% \centering
% \caption{并排子表格}
% \label{tab:subtable}
% \subfloat[第一个子表格]{
% \begin{tabular}{p{2cm}p{2cm}}
% \toprule[1.5pt]
% 111 & 222 \\\midrule[1pt]
% 222 & 333 \\\bottomrule[1.5pt]
% \end{tabular}}\hskip2cm
% \subfloat[第二个子表格]{
% \begin{tabular}{p{2cm}p{2cm}}
% \toprule[1.5pt]
% 111 & 222 \\\midrule[1pt]
% 222 & 333 \\\bottomrule[1.5pt]
% \end{tabular}}
% \end{table}

% 有的同学不想让某个表格或者图片出现在索引里面,那么请使用命令
% \verb|\caption*{}|,这个命令不会给表格编号,也就是出来的只有标题文字而没有“表
% ~XX”,“图~XX”,否则索引里面序号不连续就显得不伦不类,这也是 \LaTeX{} 里星号命
% 令默认的规则。
% 
% \begin{table}[ht]
% \centering
  % \begin{minipage}{0.45\linewidth}
  % \centering
  % \caption*{表~1.111\hskip1em 这是一个手动编号,不出现在索引中的表格。}
  % \label{tab:badtabular}
  % \begin{picture}(150,50)
    % \framebox(150,50)[c]{\zzuthesis}
  % \end{picture}    
  % \end{minipage}\hfill
  % \begin{minipage}{0.45\linewidth}
  % \centering
  % \begin{picture}(150,50)
    % \framebox(150,50)[c]{薛瑞尼}
  % \end{picture}
  % \caption*{Figure~1.111\hskip1em 这是一个手动编号,不出现在索引中的图。}
  % \label{tab:badfigure}
  % \end{minipage}
% \end{table}
% 

\subsection{小页中的脚注}

关于小页中的脚注,请看下面的例子:
 
\begin{minipage}[t]{\linewidth-2\parindent}
柳宗元,字子厚(773-819),河东(今永济县)人\footnote{山西永济水饺。},是唐
代杰出的文学家,哲学家,同时也是一位政治改革家。与韩愈共同倡导唐代古文运动,
并称韩柳\footnote{唐宋八大家之首二位。}。
\end{minipage}\\[-5pt]

\section{数学公式示例}
\label{sec:equation}

\LaTeX{} 的数学公式有两种形式:行间(inline)模式和独立(display)模式。前者是
指在正文中插入数学内容;后者独立排列,可以有或者没有编号。行间公式和无编号独立
公式都有多种输入方法,一般行间公式用 \verb|$|\ldots \verb|$|,无编号独立公式用
\verb|\[|\ldots \verb|\]|。有编号独立公式则需要用 \texttt{equation} 环境。

注意一下公式显示模式的不同,这个公式为行间模式:
$\lim_{n \to \infty} \sum_{k=1}^n \frac{1}{k^2} = \frac{\pi^2}{6}$;下面的公式
是独立模式:
\[\lim_{n \to \infty} \sum_{k=1}^n \frac{1}{k^2} = \frac{\pi^2}{6}\]

\subsection{多行公式}

\textbf{amsmath} 宏包提供了额外的行间独立(display)公式的结构,主要用于一个
公式太长一行放不下,或几个公式需要写成一组的情况,该宏包主要提供以下几个环境
:
\begin{center}
\begin{tabular}[c]{ccccccc}
equation & equation* & align & align* & gather & gather* & split \\
flalign & flalign* & multline & multline* & alignat & alignat* & \\
\end{tabular}
\end{center}

除了 \texttt{split} 外,其余环境均提供带*的版本,不生成公式编号。

% \subsection{矩阵}
% 
% 数学模式下可以用 \verb|array| 环境来生成矩阵,它提供了外部对齐和列对齐的控制参
% 数。外部对齐是指整个矩阵和周围对象的纵向关系,有三种方式:居顶、居中 (缺省) 、
% 居底,分别用 \verb|l、c、r| 来表示;列对齐也有三种方式:居左、居中、居右,分别
% 用 \verb|l、c、r| 表示。\verb|\\| 和 \verb|&| 用来分隔行和列。
% 
% \[\begin{array}{ccc}
% x_1 & x_2 & \dots \\
% x_3 & x_4 & \dots \\
% \vdots & \vdots & \ddots \\
% \end{array}\]

% \verb|amsmath| 有几个类似的环境:\verb|pmatrix、bmatrix、Bmatrix、vmatrix|~和~
% \verb|Vmatrix|,它们和 \verb|array| 的主要区别是会在表两端加上~$()\; []\;
% \{\}\; ||\; \|\|$~等分隔符,其次这些环境没有列对齐方式参数。
% 
% 行间公式可以用 \verb|smallmatrix| 环境来生成排列紧密的小矩阵,如
% $(\begin{smallmatrix} a&b\\c&d \end{smallmatrix})$。
% 

\subsubsection{长公式}

对于多行不需要对齐的长公式,我们可以用 \texttt{multline} 环境。
\begin{multline}
\framebox[.65\columnwidth]{A}\\
\framebox[.5\columnwidth]{B}\\
\shoveright{\framebox[.5\columnwidth]{C}}\\
\framebox[.65\columnwidth]{D}
\end{multline}

其代码如下:
\begin{code}
\begin{multline}
\framebox[.65\columnwidth]{A}\\
\framebox[.5\columnwidth]{B}\\
\shoveright{\framebox[.tt\columnwidth]{C}}\\
\framebox[.65\columnwidth]{D}
\end{multline}
\end{code}

\texttt{multline} 环境用于单行放不下的长公式,其第一行及最后一行分别居左、居右
对齐,两端均缩进距离 \verb|\multlinegap|;其它行则默认居中排布,但可以用个命令
\verb|\shoveleft|、\verb|\shoveright| 分别使居左或居右排布。

需要对齐的长公式可以用 \texttt{split} 环境,它本身不能单独使用,因此也称作次环境
,必须包含在 \texttt{equation} 或其它数学环境内。\texttt{split} 环境用 \verb|\\| 
和 \verb|&| 来分行和设置对齐位置。
\begin{equation}
\begin{split}
H_c&=\frac{1}{2n} \sum^n_{l=0}(-1)^{l}(n-{l})^{p-2}
\sum_{l _1+\dots+ l _p=l}\prod^p_{i=1} \binom{n_i}{l _i}\\
&\quad\cdot[(n-l)-(n_i-l_i)]^{n_i-l_i}\cdot
\Bigl[(n-l)^2-\sum^p_{j=1}(n_i-l_i)^2\Bigr].
\end{split}
\label{eqn:barwq}
\end{equation}

\subsubsection{公式组}

不需要对齐的公式组用 \texttt{gather} 环境,该环境中的公式均居中排布,各公式间用
\verb|\\| 分开;需要对齐的用 \texttt{align},在该环境中使用 \verb|\text| 命令可
以生成对单独公式的注释。
\begin{gather}
  first equation\\
  \begin{split}
    second & equation\\
           & on twolines
  \end{split}
\end{gather}

\begin{align}
 x & = y_1-y_2+y_3-y_5+y_8-\dots && \text{by \eqref{eqn:barwq}}\\
   & = y'\circ y^*               && \text{by \eqref{eqn:barwq}}\\
   & = y(0) y'                   && \text{by Axiom 1.}
\end{align}

就像单独的行间公式一样,使用 \texttt{gather}、\texttt{align} 和
\texttt{alignat} 环境生成的公式组中的每个公式也都是占据整个文本的宽度,因此
这样的公式组两侧不能再添加其它内容,比如大括号等。不过相应地用
\texttt{gathered}、\texttt{aligned} 和 \texttt{alignedat} 环境则生成仅占据
实际公式宽度的公式组。
\begin{equation*}
\left. \begin{aligned}
  \bm{B'}&=-\bm{\partial}\times \bm{E},\\
  \bm{E'}&=\bm{\partial}\times \bm{B} - 4\pi j,
\end{aligned}
\right\}
\qquad \text{Maxwell's equations}
\end{equation*}

有多种条件的公式组用 \texttt{cases} 次环境。
\[ P_{r-j}=\begin{cases}
  0& \text{if $r-j$ is odd},\\
  r!\,(-1)^{(r-j)/2}& \text{if $r-j$ is even}.
\end{cases} \]

这里仅简单介绍了 \textbf{amsmath} 的功能,更详尽的说明可参见该宏包的文档。

\subsection{定理和证明}

\zzuthesis{} 定义了常用的数学环境:
\begin{center}
\begin{tabular}{*{7}{l}}\hline
  axiom & theorem & definition & proposition & lemma & conjecture &\\
  公理 & 定理 & 定义 & 命题 & 引理 & 猜想 &\\\hline
  proof & corollary & example & exercise & assumption & remark & problem \\
  证明 & 推论 & 例子& 练习 & 假设 & 注释 & 问题\\\hline
\end{tabular}
\end{center}

以上环境的定义采用 \textbf{amsthm} 宏包提供的 \verb|\newtheorem| 命令,其语法
如下:

\begin{code}
\newtheorem{环境名}[编号延续]{显示名}[编号层次]
\end{code}

% 下面的代码定义了四个环境:定义、定理、引理和推论,它们都在一个 \verb|chapter| 
% 内编号,而引理和推论会延续定理的编号。
% \begin{code}
% \newtheorem{definition}{定义}[chapter]
% \newtheorem{theorem}{定理}[chapter]
% \newtheorem{lemma}{引理}[chapter]
% \newtheorem{corollary}[theorem]{推论}
% \end{code}
% 
% 定义了上述环境之后,我们就可以象下面这样使用它们。
下面是应用示例:
\begin{definition}
Java是一种跨平台的编程语言。
\end{definition}

\begin{theorem}
咖啡因会使人的大脑兴奋。
\end{theorem}

\begin{lemma}
茶和咖啡都会使人兴奋。
\end{lemma}

\begin{corollary}
晚上喝咖啡会导致失眠。
\end{corollary}

\texttt{proof} 环境可以用来输入下面这样的证明,它会在证明结尾输入一个~QED~符号
\footnote{拉丁语~quod erat demonstrandum~的缩写。}。

\begin{proof}[命题“物质无限可分”的证明]
一尺之棰,日取其半,万世不竭。
\end{proof}

% \begin{theorem}
% 如果函数$f(x)$满足
% \begin{enumerate}
% \item 在闭区间$[a,b]$上连续;
% \item 在开区间$(a,b)$内可导;
% \item 在区间端点处的函数值相等,即$f(a)=f(b)$,
% \end{enumerate}
% 那么在$(a,b)$内至少有一点$\xi(a<\xi<b)$,使得$f^\prime(\xi)=0$。这个定理称之为%
% \textbf{罗尔定理}。
% \label{the:rolle}
% \end{theorem}
% 
% 贝叶斯公式如式~(\ref{equ:bayes}),其中 $p(y|\mathbf{x})$ 为后验;
% $p(\mathbf{x})$ 为先验;分母 $p(\mathbf{x})$ 为归一化因子。
% \begin{equation}
% \label{equ:bayes}
% p(y|\mathbf{x}) = \frac{p(\mathbf{x},y)}{p(\mathbf{x})}=
% \frac{p(\mathbf{x}|y)p(y)}{p(\mathbf{x})} 
% \end{equation}
% 
% 再看一个 \textbf{amsmath} 的例子:
% \newcommand{\envert}[1]{\left\lvert#1\right\rvert} 
% \begin{equation}\label{detK2}
% \det\mathbf{K}(t=1,t_1,\dots,t_n)=\sum_{I\in\mathbf{n}}(-1)^{\envert{I}}
% \prod_{i\in I}t_i\prod_{j\in I}(D_j+\lambda_jt_j)\det\mathbf{A}
% ^{(\lambda)}(\overline{I}|\overline{I})=0.
% \end{equation} 
% 
% 多级规划的例子:
% \begin{equation}\label{bilevel}
% \left\{\begin{array}{l}
% \max\limits_{{\mbox{\footnotesize\boldmath $x$}}} F(x,y_1^*,y_2^*,\cdots,y_m^*)\\[0.2cm]
% \mbox{subject to:}\\[0.1cm]
% \qquad G(x)\le 0\\[0.1cm]
% \qquad(y_1^*,y_2^*,\cdots,y_m^*)\mbox{ solves problems }(i=1,2,\cdots,m)\\[0.1cm]
% \qquad\left\{\begin{array}{l}
    % \max\limits_{{\mbox{\footnotesize\boldmath $y_i$}}}f_i(x,y_1,y_2,\cdots,y_m)\\[0.2cm]
    % \mbox{subject to:}\\[0.1cm]
    % \qquad g_i(x,y_1,y_2,\cdots,y_m)\le 0.
    % \end{array}\right.
% \end{array}\right.
% \end{equation}
% 
% \begin{theorem}
% 设$f:[a,b]\rightarrow \mathbf R$为一连续函数,$g:[a,b]\rightarrow \mathbf R$为
% 一正的可积函数,那么存在一点$\xi\in [a,b]$使得:
% \[\int_a^b f(x)g(x)\,dx= f(\xi)\int_a^b g(x)\,dx\]
% \end{theorem}
% 
% \begin{proof}
% 
% 因为f是闭区间上的连续函数,f取得最大值M和最小值m。于是:
% \[mg(x)\leq f(x)g(x)\leq Mg(x)\]
% 
% 对不等式求积分,我们有:
% \[m\int_a^b g(x)\,dx\leq \int_a^b f(x)g(x)\,dx \leq M\int_a^b
% g(x)\,dx\]
% 若$\int_a^b g(x)\,dx=0$,则$\int_a^b f(x)g(x)\,dx=0$,$\xi$可取$[a,b]$上任一点
% 。
% 
% 设 $\int_a^b g(x)\,dx>0$,那么:
% \[m\leq \frac{\int_a^b f(x)g(x)\,dx}{\int_a^b g(x)\,dx}\leq M\]
% 
% 因为 $m\leq f(x)\leq M$是连续函数,则必存在一点 $\xi\in [a,b]$,使得:
% \begin{equation}
% f(\xi)= \frac{\int_a^b f(x)g(x)\,dx}{\int_a^b g(x)\,dx}
% \label{eqn:lagrange1}
% \end{equation}
% \end{proof}
% 
% \begin{corollary}
% 在式~\ref{eqn:lagrange1}中令$g(x)=1$,则可得出:
% 
% 设$f:[a,b]\rightarrow \mathbf R$为一连续函数,则$\exists\xi\in [a,b]$,使:
% \begin{equation}
% f(\xi)= \frac{\int_a^b f(x)\,dx}{b-a}
% \label{eqn:lagrange2}
% \end{equation}
% \end{corollary}
% 
% \begin{example}
  % 大家来看这个例子。
% \begin{equation}
% \label{ktc}
% \left\{\begin{array}{l}
% \nabla f({\mbox{\boldmath $x$}}^*)-\sum\limits_{j=1}^p\lambda_j\nabla g_j({\mbox{\boldmath $x$}}^*)=0\\[0.3cm]
% \lambda_jg_j({\mbox{\boldmath $x$}}^*)=0,\quad j=1,2,\cdots,p\\[0.2cm]
% \lambda_j\ge 0,\quad j=1,2,\cdots,p.
% \end{array}\right.
% \end{equation}
% \end{example}
% 
% \begin{exercise}
% 请列出 Andrew S. Tanenbaum 和 W. Richard Stevens 的所有著作。
% \end{exercise}
% 
% \begin{conjecture}
% \textit{Poincare Conjecture} If in a closed three-dimensional space, any
% closed curves can shrink to a point continuously, this space can be deformed
% to a sphere.
% \end{conjecture}
% 
% \begin{problem}
 % 回答还是不回答,是个问题。 
% \end{problem}

